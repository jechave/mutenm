\documentclass[11pt]{article}
\usepackage[utf8]{inputenc}
\usepackage[T1]{fontenc}
\usepackage{times}
\usepackage{graphicx}
\usepackage{hyperref}
\usepackage[margin=1in]{geometry}
\usepackage{natbib}
\bibliographystyle{unsrtnat}

\title{\texttt{mutenm}: An R package to mutate elastic network models of proteins}
\author{Julian Echave\\
Instituto de Ciencias Físicas (ICIFI-CONICET),\\
Universidad Nacional de San Martín,\\
Martín de Irigoyen 3100, 1650 San Martín, Buenos Aires, Argentina\\
\texttt{jechave@unsam.edu.ar}}
\date{}

\begin{document}

\maketitle

\section*{Summary}

\texttt{mutenm} is an R package for modeling mutations within the elastic network model (ENM) framework. An ENM represents a protein as a network of nodes connected by harmonic springs; despite their simplicity, ENMs capture functionally relevant collective motions through normal mode analysis.

The core function \texttt{mutenm()} takes an ENM of a wild-type protein and generates an ENM of the mutant. This enables studying how mutations affect protein structure and dynamics: comparing wild-type and mutant properties, performing mutation response scans, or simulating evolutionary trajectories through iterated mutation.

\section*{Statement of Need}

Elastic network models are widely used to study protein dynamics. Packages such as Bio3D \citep{grant2006,skjaerven2014} in R and ProDy \citep{bakan2011,zhang2021} in Python provide comprehensive ENM functionality. However, these tools focus on analyzing wild-type proteins.

The \texttt{mutenm} package extends ENM analysis to mutations. It implements the Linearly Forced ENM (LFENM) \citep{echave2008,echave2010}, which models a mutation as a perturbation to spring equilibrium lengths around the mutation site. Given a wild-type ENM, the function \texttt{mutenm()} generates a mutant ENM with altered structure. A self-consistent variant (scLFENM) \citep{echave2012} additionally recalculates the mutant's normal modes, capturing changes in dynamics as well as structure.

This approach has been used to study protein evolution---the divergence of structure across sites \citep{marcos2015,marcos2020,echave2024,echave2025}, across modes \citep{echave2008,echave2010}, and the evolution of dynamics \citep{echave2012}. The package is intended for computational biologists interested in protein structure, dynamics, and evolution.

\section*{Functionality}

\textbf{Building the ENM.} The \texttt{enm()} function constructs an elastic network model from a PDB structure. Multiple node representations are supported (C$\alpha$, C$\beta$, side-chain centroid); the optimal choice depends on the application \citep{marcos2015,echave2024}. Several force fields are available \citep{atilgan2001,yang2009,ming2005,hinsen1998,hinsen2000,moritsugu2007}; results are generally robust across models.

\textbf{Mutating the ENM.} The \texttt{mutenm()} function takes a wild-type ENM and generates a mutant ENM by perturbing spring equilibrium lengths around the mutation site. Two mutation models are available: LFENM perturbs the structure while keeping the Hessian fixed; scLFENM recalculates the Hessian after relaxation, capturing changes in dynamics. The mutant ENM can be analyzed like any ENM---comparing normal modes, covariance matrices, or flexibility between wild-type and mutant. By iterating \texttt{mutenm()}, users can simulate evolutionary trajectories.

\textbf{Mutation response scanning.} The \texttt{mrs()} function applies \texttt{mutenm()} systematically across all sites \citep{echave2021}, producing a response matrix and profiles analogous to Perturbation Response Scanning (PRS) \citep{bakan2011} but based on a mutation model rather than arbitrary forces. The response matrix (Figure~\ref{fig:mrs}) reveals how mutations at each site affect structure throughout the protein. The derived influence and sensitivity profiles identify structurally important and structurally responsive sites, which correlate with evolutionary conservation \citep{echave2025}.

\textbf{Visualization.} The \texttt{plot\_mrs()} function creates publication-ready figures showing the response matrix alongside influence and sensitivity profiles (Figure~\ref{fig:mrs}).

\begin{figure}[htbp]
\centering
\includegraphics[width=\textwidth]{paper_figure.png}
\caption{Mutation response analysis of acylphosphatase (PDB: 2ACY) using \texttt{mrs()} and \texttt{plot\_mrs()}. Left: Response matrix showing mean squared displacement at each site (y-axis) caused by mutations at each site (x-axis), with log10 color scale. Right: Influence profile (top) showing the effect of mutating each site, and sensitivity profile (bottom) showing how each site responds to mutations elsewhere.}
\label{fig:mrs}
\end{figure}

\section*{Availability}

\texttt{mutenm} is available at \url{https://github.com/jechave/mutenm} with documentation and vignettes.

\section*{Acknowledgements}

This work was supported by CONICET (grant PIP-11220210100462).

\bibliography{paper}

\end{document}
